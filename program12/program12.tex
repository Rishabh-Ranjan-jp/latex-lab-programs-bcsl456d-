\documentclass[6pt,a4paper]{report}
\usepackage[left=2cm,right=2cm,top=2cm,bottom=2cm]{geometry}
\author{Rishabh Ranjan}
\title{Block Chain Technology}
\begin{document}
	\maketitle
	\chapter {Introduction}
	\section{What is Block Chain Technology?}
	Blockchain technology is an advanced database mechanism that allows transparent information
	sharing within a business network. A blockchain database stores data in blocks that are linked together
	in a chain.
	\section{Architecture of Block chain Technology}
	Blockchain is a sequence of blocks, which holds a complete list of transaction records like
	conventional public ledger. A block has only one parent block. It is worth noting that uncle
	blocks(children of the block’s ancestors) hashes would also be stored in ethereal blockchain.
	\section{Different Types of BlockChain Technology}
	Blockchain is a sequence of blocks, which holds a complete list of transaction records like
	conventional public ledger. A block has only one parent block. It is worth noting that uncle
	blocks(children of the block’s ancestors) hashes would also be stored in ethereal blockchain.
	\chapter{Layered Architecture of Blockchain Ecosystem}
	\begin{small}
		\section{Components of BlockChain Technology}
		\begin{itemize}
			\item Node Application
			\item Distributed Database
			\item Consensus Algorithm
			\item Virtual Machine
			\item Peer-to-Peer Network
		\end{itemize}
	\section{Block Chain Layers}
\begin{enumerate}
\item Hardware/Infrastructure layer
\item Data layer
\item Network layer
\item Incentive laye
\item Contract layer
\item Application and Presentation layer
\end{enumerate}
\end{small}
\end{document}\section{LaTeX Code}
\lstinputlisting[language=TeX]{./program12.tex}
\section{LaTeX Code}
\lstinputlisting[language=TeX]{./program12.tex}
\section{LaTeX Source Code}
\lstinputlisting[language=TeX]{./program12.tex}
